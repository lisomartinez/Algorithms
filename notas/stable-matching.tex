\documentclass[12pt, fleqn]{article}
\usepackage{lipsum}
\usepackage{geometry}
\usepackage[utf8]{inputenc}
\usepackage[english]{babel}
\usepackage{amsthm}
\usepackage{amsmath}
\usepackage{amsfonts}
\usepackage{amssymb}
\usepackage{mathtools}
\usepackage{setspace}
\usepackage{minted}
\usepackage{clrscode3e}

\theoremstyle{remark}
\newtheorem*{remark}{Claim}

\theoremstyle{definition}
\newtheorem*{definition}{Claim}

\geometry{a4paper, top=30mm, bottom=35mm, left=25mm, right=25mm}
\onehalfspacing{}
\title{Stable Matching}
\author{Lisandro Martinez}
\date{\today}
%\renewcommand{\baselinestretch}{1.5}
\begin{document}
\maketitle{}
HOLA
\section{Problema}
Gale-Shapley: Dado un conjunto de preferencias entre empleadores y candidatos
es posible asignar los candidatos a los empleadores de forma tal que para
cada uno de los empleadores E y para cada candidato C que no se encuentra 
contratado para trabajar en la empresa
se cumpla al menos una de los siguientes dos casos:
\begin{enumerate}
    \item E prefiere a cada uno de los candidatos contratados en vez de C.
    \item C prefiere su actual situación antes que trabajar para E.
\end{enumerate}
    Si se cumple, el resultado es \textbf{\emph{stable}}: 
    el auto-interes de cada uno de los participantes hara que se mantenga el orden.
\paragraph{Variante a analizar:}
    Dados un conjunto de hombres V y un conjunto de mujeres M, como producir un sistema 
    en el que cada uno termine casado con su mejor opcion
\section{Definiciones}
    Un conjunto de Varones M y un conjunto de mujeres W: 
\[
\begin{split}
    M &= \{m_1,m_2, \dots m_n\} \\
    W &= \{w_1,w_2, \dots w_n\}  \\
\end{split}
\]
\begin{itemize}
    \item \(M \times W \): conjunto de todos los posibles pares ordenados de la forma
            \((m, w)\), donde \(m \in M\) y  \(w \in W\)
    \item Matching \(S\): conjunto ordenado de pares de \(M \times W\) en el que cada
             miembro de M y cada miembro de W aparece \emph{como mucho en un par de S}
    \item Perfect Matching \(S'\): es un Matching en el que cada miembro de M y cada miembro 
            de W aparecen \emph{exactamente en un par de \(S'\)}\\
          Un perfect matching es una manera de sencilla de emparejar los hombres con las 
            mujeres, de modo tal que cada uno termine casado con alguien y nadie se case 
            con mas de una persona.
    \item Preferences: cada \(m \in M\) establece un ranking de mujeres con las que
            querria casarse en orden decreciente. 
    \item Instability: Existen dos pares \(m,w\) y \(m', w'\) en \(S\) en los que 
            \(m\) prefiere a \(w'\) por sobre \(w\) y \(w\) prefiere a \(m\) por sobre \(m'\).
          En este caso no existiría nada que les impida a \(m\) y a \(w'\) abandonar a sus 
          respectivas parejas y comprometerse entre ellos. El par \(m,w'\) es una
           inestabilidad con respecto de \(S\), dado que no pertenece a S, 
           pero \(m\) y a \(w'\) se prefieren entre si antes que a sus respectivas parejas.
    \item Stable Matching: un matching es estable si \(i\) es perfecto y \(ii\) no existe
            ninguna inestabilidad con respecto a \(S\). 
            \emph{\textbf{Nota:} es posible que exista mas de un stable matching para una 
                instancia del problema}         
\end{itemize}
\section{Diseno del algoritmos}
\begin{itemize}
    \item Existe un stable matching para cada conjunto de listas de preferencias entre
            hombres y mujeres.
    \item Es posible construir un algoritmo eficiente que toma como entrada 
    la lista de preferencias y produce un stable matching.
\end{itemize}
\subsection{Observaciones}
\begin{itemize}
    \item Al comienzo, tanto los hombres como las mujeres se encuentran solteros.
            un hombre soltero elije a la mujer que se encuentra primera en su ranking
            y le hace una propuesta de casamiento.
            La mujer acepta la proposición y los dos pasan a estar comprometidos.
    \item En algún momento de la ejecución del algoritmo algunos hombres y mujeres
            se encuentran comprometidos y otros solteros. El próximo paso consiste en:
            \begin{enumerate}
                \item El hombre \(m\) le propone casamiento a la mujer que se encuentra
                        primera en su lista de preferencias.
                \item Si la mujer \(w\) esta soltera, acepta la proposición y se comprometen
                        Si, por contraposición, la mujer \(w\) ya se encuentra comprometida
                        con \(m'\) puede ocurrir una de las siguientes opciones:
                        \begin{enumerate}
                            \item \(w\) prefiere a \(m'\) por sobre \(m\), \(w\) rechaza
                                    a \(m\) y continúa comprometida con \(m'\).
                            \item \(w\) prefiere a \(m\) por sobre \(m'\), \(w\) se
                                    se compromete con \(m\) y \(m'\) queda soltero.        
                        \end{enumerate}
            \end{enumerate}
\end{itemize}
\subsection{Algoritmo Gale-Shapley}
\begin{verbatim}
    Initialize each person to be free.

    while (some man is free and hasn't proposed to every woman) {
        Choose such a man m
        w = 1st woman on m's list to whom m has not yet proposed
        if (w is free)
            assign m and w to be engaged
        else if (w prefers m to her fiancé m')
            assign m and w to be engaged, and m' to be free
        else
            w rejects m
    }
\end{verbatim}
\section{Análisis del algoritmo}
\subsection{Demostración de la finalización del algoritmo}
\begin{itemize}
    \item (1,1): \(w\) permanece comprometida desde el momento en el que recibe la primera
            propuesta. Solo cambia de pretendiente si \(m\) se encuentra en mejor 
            posición que \(m'\) en su lista de preferencias.
    \item (1,2): \(m\) le realiza propuestas de casamiento a las mujeres de su 
            lista de preferencias de mejor a peor. 
\end{itemize}
\begin{definition}
    El algoritmo G-S termina después de un máximo de \(n^2\) iteraciones del \emph{while}.
\end{definition}
\begin{proof}
    Se utiliza las proposiciones de matrimonio como medida de \emph{progreso}.
\end{proof}
\end{document}
